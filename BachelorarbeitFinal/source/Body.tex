\section{Einleitung}
%for reference to this section
\label{section:Introduction}


\subsection{Forschungsfrage}
%for references to this subsection
\label{subsection:Coding}


\subsection{Aufbau}
derzeitiges Literaturverzeichnis:\newline
\autocite[]{Bracey.2014}\newline
\autocite[]{Coyier.2012}\newline
\autocite[]{Croom.2012}\newline
\autocite[]{Firdaus.}\newline
\autocite[]{Hixon.2011}\newline
\autocite[]{Page.2013}\newline
\autocite[]{ZingDesign.2014}\newline


\section{Generell was bedeutet stylen einer Website, wofür gibt es css pre-processoren}
\label{section:MathematicalStuff}
hier vielleicht auch auf die Hilfsprogramme zum rendern von CSS eingehen. (Bsp. Compass oder Sassmeister)

\selectlanguage{ngerman}

\subsection{CSS 3}
bei den Unterkategorien wird auf Punkte wie Installation, generelle Verwendung mit kleinen Beispielen und Vor- und Nachteile der einzelnen Sprachen eingegangen.

\subsection{Less}

\subsection{Sass}
hier eventuell auch kurz auf haml eingehen.
Unterschiede zwischen Sass und Scss erklären. 
\subsection{Stylus}

\section{Implementierung}
Hier wird denke ich genauer erklärt, wie die Tools bei dem, für den pratkischen Teil verwendeten, Projekt implemetiert werden müssen und welche  Vorraussetzungen das Projekt besitzen muss um die jeweilige Sprache verwenden zu können.

\subsection{CSS 3}

\subsection{Less}

\subsection{Sass}

\subsection{Stylus}


\section{Praktische Anwendung und Vergleiche}
Beim praktischen Teil wird dokumentiert, wie das verwendete Projekt aufgebaut ist und wie es mit der jeweiligen Sprache designed wurde. Es werden Abschnitte aus dem Code genau analysiert. Umd das gut umsetzen zu können, wird das Projekt, also die Webseite wirklich von grund auf in CSS3, Less, Sass und Stylus komplett eigens gestylt. So soll auch gut eruiert werden können, welche Sprache welche Vor- und Nachteile hat und wie erheblich diese in der Programmierung zu spüren sind. 

\section{Ergebnisse}
hier möchte ich die Ergebnisse aus dem praktischen Teil zusammenfassen: Installation(dauer, Schwierigkeiten ...), Codequalität, Codekomplexität, Browserkompatibilität (vielleicht fällt mir bei der Recherche noch mehr ein, was ich vorher vergleiche kann und hier präsentieren)

\section{Schluss}
ist eh klar, was hier kommt ;)




