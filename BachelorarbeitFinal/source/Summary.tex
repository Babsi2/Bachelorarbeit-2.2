\section*{Kurzfassung}
\begin{tabular}{l l}
Vor- und Zuname:& Barbara HUBER\\
Institution: & FH Salzburg\\ 
Studiengang: &  Bachelor MultiMediaTechnology\\ 
Titel der Bachelorarbeit: & Sass vs. Less vs. Stylus\\
Begutachter: & Hannes Moser\\ 
\end{tabular} 
\vspace{0.5cm}

Die folgende Arbeit beschäftigt sich mit der Verwendung von CSS-Präprozessoren, im speziellen mit LESS, Sass und Stylus. Es wird beschrieben, was Präprozessoren im Allgmeinen sind und wie diese angewendet werden. Um die Verwendung der Präprozessoren verständlicher erklären zu können, wird in der Arbeit auch auf den Begriff Workflow eingegangen und erörtert, wie der Workflow mithilfe von Präprozessoren beeinflusst wird. \newline
Um zu verstehen, wie Präprozessoren arbeiten, ist es notwendig zu wissen, was ein Parser ist und was unter Software Komponenten zu verstehen ist. Zwei Kapitel der Arbeit beschäftigen sich mit diesen Themen und erläutern anhand von Beispielen und den Präprozessoren LESS, SCSS bzw. Sass und Stylus, wie die jeweiligen Parser der Präprozessoren im Detail funktionieren. Auch das Kapitel zu Software Komponenten wird in Zusammenhang mit den erwähnten Präprozessoren erklärt.\newline
Wie zu Beginn erwähnt, befasst sich die Arbeit im Speziellen mit den CSS-Präprozessoren LESS, Sass und Stylus. \newline
Nach der Explikation von Präprozessoren, Parser, Workflow und Software Komponenten werden die, der Arbeit zu Grunde liegenden Präprozessoren im Detail beschrieben. \newline
Nach der Theoretischen Darlegung des Themas, wird die praktische Arbeit beschrieben und erläutert, wie diese umgesetzt wurde. Im Anschluss daran, wird im letzten Kapitel der Arbeit diskutiert, welche Ergebnisse der praktische Teil der Arbeit ergeben hat und welche Relevanz das Thema und die Inhalte der Arbeit auf weitere Forschungsfragen bzw. Problemaspekte hat.
\paragraph{Schlagwörter: Präprozessoren, Workflow, Software Komponenten, CSS, LESS, SCSS, Sass, Stylus}

